
% Absolute Galois group

1. the absolute Galois group G_K of a field K is Gal(K^{sep}/K) 
i.e. the Galois group of K^{sep} over K, where K^sep is a separable closure of K.

2. When K is a perfect field, K^{sep} is the same as an algebraic closure K^{alg} of K. 
This holds e.g. for K of characteristic zero, or K a finite field. 

3. The absolute Galois group of a finite field K is isomorphic to the group
\hat{\mathbf{Z}} := lim_\larr \mathbf{Z}/n\mathbf{Z}. 

4. G_\Q := Gal(\bar\Q/\Q), where \bar\Q or \bar Q is the algebraic closure of 
the field of rational numbers i.e. the field of algebraic numbers.

\F_1

\Z_p \defeq lim_{\larr}\Z/p^n\Z




% algebraic structure

\definition-\lemma. Let $F$ be a subfield of a field $L$. An element $a \in L$ is called 
algebraic over $F$ if one of the following equivalent conditions is satisfied 
  (1) $f(a) = 0$ for a non-zero polynomial $f(X) \in F[X]$
  (2) elements $1, a, a^2, \cdots$ are linearly dependent over $F$
  (3) F-vector space $F[a] = \{ \sum a_ka^k : a_k \in F \}$ is of finite dimension over $F$
  (4) $F[a] = F(a)$

\proof. Omitted.

  For an element a algebraic over $F$ denote by
                                $f_a(X) \in F[X]$
the monic polynomial of minimal degree such that $f_a(a) = 0$. 
  This polynomial is irreducible. $f_a(X)$ is a linear polynomial iff $a \in F$.

\lemma. Define a ring homomorphism $F[X] \rarr L$, $g(X) \rarr g(a)$. Its kernel is the
principal ideal generated by $f_a(X)$ and its image is $F(a)$, so
                              $F[X]/(f_a(X)) \cong F(a)$
\proof. Using the division algorithm.
                            
\definition. A field $L$ is called algebraic over its subfield $F$ if every element of $L$ is
algebraic over $F$. The extension $L/F$ is called algebraic. 

\definition. Let $F$ be a subfield of a field $L$. The dimension of $L$ as a vector space
over $F$ is called the degree $[L : F]$ of the extension $L/F$.

  If a is algebraic over $F$ then $[F(a) : F]$ is finite and it equals the degree of the
monic irreducible polynomial $f_a$ of a over $F$.



% field


\S 1. Finite fields. Let F be a field (commutative or not) with the
unit-element 1. Its characteristic must clearly be a prime p > 1, 
prime ring in F is isomorphic to the prime field \mathbf F_p = \mathbf Z/p\mathbf Z, with 
which we may identify it. Then F may be regarded as a vector-space over \mathbf F_p;  
as such, it has an obviously finite dimension n, and the number of its 
elements is q = p^n. 

\theorem (Wedderburn). All finite fields are commutative. 
\proof. Omitted.

\lemma. If K is a commutative field, every finite subgroup of K^\times is cyclic.

\theorem. Let K be an algebraically closed field of characteristic p > 1. 
Then, for every f \ge 1, K contains one and only one field F = F_q with 
q = p^n elements; F consists of the roots of X^q = X in K; F^\times consists 
of the roots of X^{q-1} = 1 in K and is a cyclic group of order q-1. 

\begin{proof}
  If F is any field with q elements, lemma shows that F^\times is a cyclic 
group of order q-1. Thus, if K contains such a field F, F^\times must consist 
of the roots of X^{q-1} = 1, hence F of the roots of X(X^{q-1}-1) = X^q-X = 0, 
so that both are uniquely determined. 
  Conversely, if q = p^n, x \mapsto x^q is an automorphism of K, 
so that the elements of K which are fixed under it make up a field F 
consisting of the roots of X^q - X = 0; as it is clear that X^q - X 
has only simple roots in K, F is a field with q elements. 
\end{proof}

\corollary. Up to isomorphisms, there is one and only one field with q = p^n elements. 
\proof. Omitted.






% modular


\S 1. The periodicity of classical modular function j can be 
described algebraically by saying that j(\tau) = j(\frac{a\tau+b}{c\tau+d}) 
for any a, b, c, d \in \Z with ad - bc = 1.

\theorem (Kronecker). for any integer \tau in the quadratic field \Q(\sqrt{-D}), 
j(\tau) is an algebraic integer whose degree is the class number of \Q(\sqrt{-D})

\example. for \Q(\sqrt{-1}), class number = 1, j(-1) = 12^3.
\example. for \Q(\sqrt{-163}), class number = 1, j((1+\sqrt{-163})/2) = -640320^3.





% group scheme

\mathbb G_m := Spec(\Z[U, U^-1]), 
\mathbb G_m \times_\Z \C, 



