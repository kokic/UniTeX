
  \S 1. Finite fields. Let F be a field (commutative or not) with the
unit-element 1. Its characteristic must clearly be a prime p > 1, 
prime ring in F is isomorphic to the prime field \mathbf F_p = \mathbf Z/p\mathbf Z, with 
which we may identify it. Then F may be regarded as a vector-space over \mathbf F_p;  
as such, it has an obviously finite dimension n, and the number of its 
elements is q = p^n. 

\theorem (Wedderburn). All finite fields are commutative. 
\proof. Omitted.

\lemma. If K is a commutative field, every finite subgroup of K^\times is cyclic.

\theorem. Let K be an algebraically closed field of characteristic p > 1. 
Then, for every f \ge 1, K contains one and only one field F = F_q with 
q = p^n elements; F consists of the roots of X^q = X in K; F^\times consists 
of the roots of X^{q-1} = 1 in K and is a cyclic group of order q-1. 

\begin{proof}
  If F is any field with q elements, lemma shows that F^\times is a cyclic 
group of order q-1. Thus, if K contains such a field F, F^\times must consist 
of the roots of X^{q-1} = 1, hence F of the roots of X(X^{q-1}-1) = X^q-X = 0, 
so that both are uniquely determined. 
  Conversely, if q = p^n, x \mapsto x^q is an automorphism of K, 
so that the elements of K which are fixed under it make up a field F 
consisting of the roots of X^q - X = 0; as it is clear that X^q - X 
has only simple roots in K, F is a field with q elements. 
\end{proof}

\corollary. Up to isomorphisms, there is one and only one field with q = p^n elements. 
\proof. Omitted.
